\section{Introducci'on}

En este trabajo pr'actico dise\~{n}amos e implementamos una base de datos para el Registro 'Unico de Accidentes de Tr'ansito (RUAT), sistema que est'a que est'a preparando el Gobierno Nacional, para registrar y analizar informaci'on sobre accidentes e infracciones de tr'ansito ocurridos en el pa'is.

Por un lado, el sistema registra todos los datos relacionados con siniestros de tr'ansito, lo cual abarca:
\begin{itemize}
\item veh'iculos involucrados;
\item conductores involucrados;
\item testigos;
\item localizaci'on;
\item modalidad del siniestro (atropello, vuelco, etc.);
\item tipo de colisi'on;
\item denuncia radicada por el hecho;
\item estudios y peritajes hechos.
\end{itemize}

Por otro lado, registra infracciones de tr'ansito, y m'as espec'ificamente:

\begin{itemize}
\item veh'iculo involucrado;
\item conductor involucrado;
\item localizaci'on;
\item tipo de infracci'on;
\end{itemize}

Adem'as, el sistema registra datos sobre los veh'iculos, personas y las v'ias nacionales. Sobre los veh'iculos, almacena:

\begin{itemize}
\item categor'ia de coche (gama media, gama alta, etc.);
\item tipo de veh'iculo (auto, cami'on, moto, etc.);
\item seguro automotor y su tipo;
\end{itemize}

Sobre las personas, almacena:

\begin{itemize}
\item datos personales;
\item autos de los cuales es due\~{n}a;
\item cedulas (verdes y azules) que posee;
\item licencias de conducir que posee;
\item antecedentes penales;
\end{itemize}

Finalmente, sobre las v'ias nacionales, el sistema almacena:

\begin{itemize}
\item tipo de v'ia seg'un el tramo (calle, avenida, autopista, etc.);
\item extensi'on del tramo;
\end{itemize}

La elicitaci'on de estos requerimientos proviene, principalmente, de la lectura del enunciado del trabajo pr'actico, que contiene toda esta informaci'on.