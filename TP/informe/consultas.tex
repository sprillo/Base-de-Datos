\section{Consultas SQL}

Dos eran las funcionalidades que esperaban poder realizarse. La primera de ellas ped'ia lo siguiente:

\textit{Obtener, con un número de licencia específico,
información sobre los accidentes en los que ha participado el conductor propietario de
la misma, con detalles de fecha, lugar, tipo de accidente, participación y modalidad.
También se deberá indicar la cantidad de automóviles que está habilitado a conducir.}

La consulta SQL correspondiente es la siguiente:

\begin{verbatim}
SELECT s.fecha, c.nombre, d.altura, lo.nombre, pr.nombre, m.tipo, tc.tipo, svp.culpable, 
        au.cantidad_autos
FROM siniestro_vehiculo_persona svp, 
        persona p, 
        persona_con_licencia pl, 
        licencia l, 
        siniestro s,
        siniestro_forma_de_modalidad f,
        modalidad m,
        direccion d,
        calle c,
        localidad lo, 
        provincia pr,
        siniestro_damnifica_tipo_de_colision dam, 
        tipo_de_colision tc,
        (SELECT COUNT(*) cantidad_autos
            FROM licencia l2 NATURAL JOIN 
                persona_con_licencia pl2 NATURAL JOIN
                cedula NATURAL JOIN
                vehiculo
            WHERE l2.nroLicencia = 0 -- la licencia del Chano
        ) au
WHERE l.nroLicencia = 0		-- la licencia del Chano
-- join entre licencia, persona_con_licencia, persona y siniestro_vehiculo_persona
AND pl.dni = l.dni
AND p.dni = pl.dni
AND svp.dni = p.dni
-- join entre modalidad, siniestro_forma_de_modalidad, siniestro y siniestro_vehiculo_persona
AND s.idSiniestro = svp.idSiniestro
AND f.idSiniestro = s.idSiniestro
AND f.idTipoModalidad = m.idTipoModalidad
-- join entre siniestro, siniestro_damnifica_tipo_de_colision y tipo_de_colision
AND s.idSiniestro = dam.idSiniestro
AND dam.idTipoColision = tc.idTipoColision
-- join entre siniestro, direccion, calle, localidad y provincia
AND s.altura = d.altura
AND s.idCalle = d.idCalle
AND s.idLocalidad = d.idLocalidad
AND s.idProvincia = d.idProvincia
-- join entre direccion y calle
AND d.idCalle = c.idCalle
AND d.idLocalidad = c.idLocalidad
AND d.idProvincia = c.idProvincia
-- join entre calle y localidad
AND c.idLocalidad = lo.idLocalidad
AND c.idProvincia = lo.idProvincia
-- join entre localidad y provincia
AND lo.idProvincia = pr.idProvincia;
\end{verbatim}

El segundo requerimiento era:

\textit{Obtener, dada una modalidad de accidente (atropello, vuelco, incendio, caída del ocupante, etc), un listado de licencias de conducir y la cantidad de veces que cada una de estas licencias incurrió en la modalidad consultada.}

En esta oportunidad, decidimos abreviar la consulta utilizando el comando \texttt{JOIN}, en lugar de hacer las juntas manualmente. La consulta que resuelve el problema es:

\begin{verbatim}
SELECT m.idTipoModalidad, j.nroLicencia, j.dni, COUNT(*)
FROM (modalidad m NATURAL JOIN 
        siniestro_forma_de_modalidad sm NATURAL JOIN 
        siniestro_vehiculo_persona svp NATURAL JOIN 
        persona_con_licencia pl NATURAL JOIN 
        licencia l) j
WHERE m.idTipoModalidad = 0
GROUP BY j.nroLicencia;
\end{verbatim}