\section{Conclusi'on}

En este trabajo hemos estudiado dos modelos de bases de datos no relacionales.

Por un lado, hemos visto, en profundidad, c'omo realizar un modelado de una \textit{document oriented}, responder consultas sobre la base de datos, realizar consultas MapReduce, y aplicar t'ecnicas de sharding para que los accesos a la base de datos en un hipot'etico escenario de clustering, sea eficiente. Se pudo ver que este el modelado en esta clase de bases de datos es sencillo, y permite que las consultas sean m'as sencillas que en SQL tradicional, puesto que este dise\~no se orienta espec'ificamente a responder las consultas de nuestro inter'es. La capacidad de realizar consultas MapReduce y de utilizar sharding ponen en evidencia lo preparado de esta clase de base de datos para un escenario de c'omputo distribu'ido, en el cual los datos se distribuyen sobre m'as de una computadora. Bajo ciertas hip'otesis de uniformidad sobre los datos ingresados en la base de datos, mostramos que el sharding permite balancear la carga de datos sobre todos los shards, lo cual, en el mundo real, tiene un impacto no s'olo en la optimizaci'on de los recursos de hardware (ning'un nodo requiere demasiado disco), sino tambi'en desde el punto de vista del tiempo de respuesta de las queries (el ancho de banda necesario para mantener un delay aceptable es bajo, pues no hay nodos que sean cuellos de botella).

Por otro lado, estudiamos superficialmente el modelo \textit{column oriented}. Como se vi'o, este modelo provee prestaciones similares, al permitir tanto consultas MapReduce como sharding. La disyuntiva en la elecci'on entre document oriented y column oriented depender'a fuertemente del contexto en el que estemos trabajando.