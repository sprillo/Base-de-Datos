\section{Introducci'on}

En muchos casos en los que debemos trabajar con grandes vol'umenes de datos, SQL no resulta una opci'on factible. La principal ventaja de las bases de datos NoSQL es su posibilidad de escalar en forma horizontal (adem'as de vertical). En otras palabras, para incrementar el vol'umen de datos que una base SQL tradicional puede procesar, la 'unica alternativa es utilizar una \emph{mejor} computadora. Por el contrario, al estar trabajando con una base de datos NoSQL podemos hacer crecer el flujo de datos manejado a trav'es de una \emph{mayor cantidad} de computadoras. Esto significa que en entornos en los que la cantidad de datos crece muy r'apidamente (y muchas veces no para de crecer nunca), debemos optar por un crecimiento distribu'ido de nuestra arquitectura, lo cual es 'unicamente soportado por una base de datos NoSQL.

En este trabajo exploramos dos tipos de bases de datos NoSQL: \textit{document oriented} y \textit{column family}. La primera es explorada exaustivamente, tratando su dise\~no, consultas y su soporte de MapReduce. Exploramos tambi'en la efectividad de la t'ecnica de \textit{sharding} sobre estas bases. De la segunda realizamos un an'alisis m'as superficial,  observando c'omo resolver'ia cada uno de los problemas tratados para \textit{document oriented}.

