\section{Ejercicio 1}

En este ejercicio dise\~namos una base de datos NoSQL de tipo documentos con la capacidad de responder rapidamente a varias consultas. Para ello, empleamos desnormalizaci'on, dise\~nando documentos adecuados para las consultas.

\subsection{Desnormalizaci'on}

Desnormalizamos el esquema, de la forma tradicional. Cada entidad tiene un documento asociado. Las relaciones 1:N se incluyen en el documento asociado al lado 1. Las relaciones N:M se incluyen a ambos lados de la relaci'on.

\subsubsection{Documento \texttt{empleados}}

\begin{verbatim}
empleados {
    nro_legajo : INTEGER
    nombre : STRING
    clientes_atendidos : [{dni : STRING, edad : INTEGER, fecha : DATETIME}]
    sectores_donde_trabaja : [{cod_sector : INTEGER, id_tarea : INTEGER}]
}
\end{verbatim}

Observar que en \texttt{clientes\_atendidos} nos guardamos, adem'as del atributo identificatorio \texttt{dni} y el atributo de relaci'on \texttt{fecha}, la \texttt{edad} de los clientes atendidos. Este atributo ser'a necesario para responder una de las consultas, posteriormente. 

Insertamos algunos documentos:

\begin{verbatim}
db.empleados.insert({
    nro_legajo : 1,
    nombre : "empleado1",
    clientes_atendidos : [	{dni : "11111111", edad : 18, fecha : "01/01/2015"},
                           {dni : "22222222", edad : 17, fecha : "01/02/2015"}],
    sectores_donde_trabaja : [{cod_sector : 1, id_tarea : 1} , 	{cod_sector : 2, cod_tarea : 1}]
})

db.empleados.insert({
    nro_legajo : 2,
    nombre : "empleado2",
    clientes_atendidos : [ {dni : "22222222", edad : 17, fecha : "02/02/2015"} ],
    sectores_donde_trabaja : [{cod_sector : 1, id_tarea : 1}, 	{cod_sector : 2, cod_tarea : 1}]
})

db.empleados.insert({
    nro_legajo : 3,
    nombre : "empleado3",
    clientes_atendidos : [{dni : "33333333", edad : 19, fecha : "01/03/2015"}],
    sectores_donde_trabaja : [{cod_sector : 1, id_tarea : 1}, 	{cod_sector : 2, cod_tarea : 1}]
})

db.empleados.insert({
    nro_legajo : 4,
    nombre : "empleado4",
    clientes_atendidos : [],
    sectores_donde_trabaja : [{cod_sector : 1, id_tarea : 1}]
})
\end{verbatim}

\subsubsection{Documento \texttt{articulos}}

\begin{verbatim}
articulos {
    codigo : STRING
    nombre : STRING
    ventas : [{dni : STRING}]
}
\end{verbatim}

Insertamos algunos documentos:

\begin{verbatim}
db.articulos.insert({
    codigo : "1",
    nombre : "articulo1",
    ventas : [{dni :	 "11111111"}, {dni : "22222222"}]
})

db.articulos.insert({
    codigo : "2",
    nombre : "articulo2",
    ventas : [{dni : "11111111"}, {dni : "33333333"}]
})

db.articulos.insert({
    codigo : "3",
    nombre : "articulo3",
    ventas : [{dni : "33333333"}]
})
\end{verbatim}

\subsubsection{Documento \texttt{sectores}}

\begin{verbatim}
sectores{
    cod_sector : INTEGER
    articulos : [{codigo : STRING}]
    trabaja : [{nro_legajo : INTEGER, id_tarea : INTEGER}]
}
\end{verbatim}

Insertamos algunos documentos:

\begin{verbatim}
db.sectores.insert({
    cod_sector : 1,
    articulos : [{codigo : "1"}],
    trabaja : [	{nro_legajo : 1, id_tarea : 1},
               {nro_legajo : 2, id_tarea : 1},
               {nro_legajo : 3, id_tarea : 1},
               {nro_legajo : 4, id_tarea : 1}]
})

db.sectores.insert({
    cod_sector : 2,
    articulos : [{codigo : "2"}],
    trabaja : [	{nro_legajo : 1, id_tarea : 1},
               {nro_legajo : 2, id_tarea : 1},
               {nro_legajo : 3, id_tarea : 1}]
})
\end{verbatim}

\subsubsection{Documento \texttt{clientes}}

\begin{verbatim}
clientes{
    dni : STRING
    nombre : STRING,
    edad : INTEGER,
    atendido_por : [{nro_legajo : INTEGER, fecha : DATETIME}],
    compro_articulos : [{codigo : INTEGER}]
}
\end{verbatim}

Insertamos algunos documentos:

\begin{verbatim}
db.clientes.insert({
    dni : "11111111",
    nombre : "cliente1",
    edad : 18,
    atendido_por : [{nro_legajo : 1, fecha : "01/01/2015"}],
    compro_articulos : [{codigo : "1"} , {codigo : "2"}]
})

db.clientes.insert({
    dni : "22222222",
    nombre : "cliente2",
    edad : 17,
    atendido_por : [{nro_legajo : 1, fecha : "01/02/2015"} , {nro_legajo : 2, 
                                                              fecha : "02/02/2015"}],
    compro_articulos : [{codigo : "1"}]
})

db.clientes.insert({
    dni : "33333333",
    nombre : "cliente3",
    edad : 19,
    atendido_por : [{nro_legajo : 1, fecha : "01/03/2015"}],
    compro_articulos : [{codigo : "2"} , {codigo : "3"}]
})
\end{verbatim}

\subsection{Consultas}

\subsubsection{Los empleados que atendieron clientes mayores de edad}

\begin{verbatim}
db.empleados.find(
    {"clientes_atendidos.edad" : {$gte : 18}},
    {nro_legajo : 1, nombre : 1}
)
\end{verbatim}

Equivalenemente, podriamos pedir:

\begin{verbatim}
db.empleados.find(
    {"clientes_atendidos" : {$elemMatch : {"edad" : {$gte : 18}}}},
    {nro_legajo : 1, nombre : 1}
)
\end{verbatim}

La query nos retorna, como deseamos:

\begin{verbatim}
{ "_id" : ObjectId("56420cfd92d300ad159887da"), "nro_legajo" : 1, "nombre" : "empleado1" }
{ "_id" : ObjectId("56420cfd92d300ad159887dc"), "nro_legajo" : 3, "nombre" : "empleado3" }
\end{verbatim}

\subsubsection{Los art'iculos m'as vendidos}

\begin{verbatim}
db.articulos.aggregate([
    {$project : {"cant_ventas" : {$size : "$ventas"} , nombre : 1, codigo : 1}},
    {$sort : {"cant_ventas" : -1}},
])
\end{verbatim}

La respuesta a esta consulta es:

\begin{verbatim}
{ "_id" : ObjectId("56420d1e92d300ad159887de"), "codigo" : "1", "nombre" : "articulo1",
    "cant_ventas" : 2 }
{ "_id" : ObjectId("56420d1e92d300ad159887df"), "codigo" : "2", "nombre" : "articulo2",
    "cant_ventas" : 2 }
{ "_id" : ObjectId("56420d1e92d300ad159887e0"), "codigo" : "3", "nombre" : "articulo3",
    "cant_ventas" : 1 }
\end{verbatim}

As'i, vemos que los articulos mas vendidos son los de codigo 1 y 2, cada uno con 2 ventas.


\subsubsection{Los sectores donde trabajan exactamente 3 empleados}

\begin{verbatim}
db.sectores.find(
    {trabaja : {$size : 3}},
    {cod_sector : 1}
)
\end{verbatim}

La respuesta a esta consulta es:

\begin{verbatim}
{ "_id" : ObjectId("56420d9e92d300ad159887e2"), "cod_sector" : 2 }
\end{verbatim}

Efectivamente, el sector 2 contiene exactamente 3 trabajadores.


\subsubsection{El empleado que trabaja en m'as sectores}

\begin{verbatim}
db.empleados.aggregate([
    {$project : {cant_sectores : {$size : "$sectores_donde_trabaja"} , nombre : 1, nro_legajo : 1}},
    {$sort : {cant_sectores : -1}},
    {$limit : 1}
])
\end{verbatim}

La respuesta a esta consulta es:

\begin{verbatim}
{ "_id" : ObjectId("56420cfd92d300ad159887da"), "nro_legajo" : 1, "nombre" : "empleado1",
    "cant_sectores" : 2 }
\end{verbatim}

Si bien los empleados 2 y 3 tambien trabajan en dos sectores, solo se nos pide dar uno solo, asi que desempatamos arbitrariamente.

\subsubsection{Ranking de los clientes con mayor cantidad de compras}

\begin{verbatim}
db.clientes.aggregate([
    {$project : {cant_compras : {$size : "$compro_articulos"} , nombre : 1, dni : 1}},
    {$sort : {cant_compras : -1}}
])
\end{verbatim}

La respuesta a esta consulta es:

\begin{verbatim}
{ "_id" : ObjectId("56420df292d300ad159887e3"), "dni" : "11111111", "nombre" : "cliente1",
    "cant_compras" : 2 }
{ "_id" : ObjectId("56420df392d300ad159887e5"), "dni" : "33333333", "nombre" : "cliente3",
    "cant_compras" : 2 }
{ "_id" : ObjectId("56420df292d300ad159887e4"), "dni" : "22222222", "nombre" : "cliente2",
    "cant_compras" : 1 }
\end{verbatim}

Como vemos, los clientes aparecen ordenamos de mayor a menos cantidad de compras.

\subsubsection{Cantidad de compras realizadas por clientes de misma edad}

\begin{verbatim}
db.clientes.aggregate([
    {$project : {cant_compras : {$size : "$compro_articulos"} , nombre : 1, dni : 1, edad : 1}},
    {$group :{
        _id : "$edad",
        total_compras : {$sum : "$cant_compras"}
    }}
])
\end{verbatim}

El resultado de la query es:

\begin{verbatim}
{ "_id" : 19, "total_compras" : 2 }
{ "_id" : 17, "total_compras" : 1 }
{ "_id" : 18, "total_compras" : 2 }
\end{verbatim}

En definitiva, este es un diseño de documentos que, mediante documentos adecuados y redundancia, permite responder rapidamente a todas las queries pedidas.
