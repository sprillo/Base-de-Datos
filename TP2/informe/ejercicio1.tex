\section{Ejercicio 1}

En este ejercicio diseñamos una base de datos NoSQL de tipo documentos con la capacidad de responder rapidamente a varias consultas. Para ello, empleamos desnormalizacion, diseñando documentos adecuados para las consultas.

\subsection{Query 1}
Debemos poder responder rapidamente cuales son los empleados que atendieron clientes mayores de edad. Para ello, tendremos una tabla \textbf{empleados}, con los siguientes campos:

\begin{verbatim}
empleados : {
	nro_legajo : INTEGER
	nombre : STRING
	clientes_atendidos : [{dni : STRING, edad : INTEGER, fecha : DATETIME}]
	sectores_donde_trabaja : [{cod_sector : INTEGER, id_tarea : INTEGER}]
}
\end{verbatim}

Insertemos algunos empleado a la tabla:

\begin{verbatim}
db.empleados.insert({
	nro_legajo : 1,
	nombre : "empleado1",
	clientes_atendidos : [	{dni : "11111111", edad : 18, fecha : "01/01/2015"},
							{dni : "22222222", edad : 17, fecha : "01/02/2015"}],
	sectores_donde_trabaja : [{cod_sector : 1, id_tarea : 1} , 	{cod_sector : 2, cod_tarea : 1}]
})

db.empleados.insert({
	nro_legajo : 2,
	nombre : "empleado2",
	clientes_atendidos : [ {dni : "22222222", edad : 17, fecha : "02/02/2015"} ],
	sectores_donde_trabaja : [{cod_sector : 1, id_tarea : 1}, 	{cod_sector : 2, cod_tarea : 1}]
})

db.empleados.insert({
	nro_legajo : 3,
	nombre : "empleado3",
	clientes_atendidos : [{dni : "33333333", edad : 19, fecha : "01/03/2015"}],
	sectores_donde_trabaja : [{cod_sector : 1, id_tarea : 1}, 	{cod_sector : 2, cod_tarea : 1}]
})

db.empleados.insert({
	nro_legajo : 4,
	nombre : "empleado4",
	clientes_atendidos : [],
	sectores_donde_trabaja : [{cod_sector : 1, id_tarea : 1}]
})
\end{verbatim}

De esta manera, la query se responde rapidamente con la siguiente consulta:

\begin{verbatim}
db.empleados.find(
	{"clientes_atendidos.edad" : {$gte : 18}},
	{nro_legajo : 1, nombre : 1}
)
\end{verbatim}

Equivalenemente, podriamos pedir:

\begin{verbatim}
db.empleados.find(
	{"clientes_atendidos" : {$elemMatch : {"edad" : {$gte : 18}}}},
	{nro_legajo : 1, nombre : 1}
)
\end{verbatim}

La query nos retorna, como deseamos:

\begin{verbatim}
{ "_id" : ObjectId("56420cfd92d300ad159887da"), "nro_legajo" : 1, "nombre" : "empleado1" }
{ "_id" : ObjectId("56420cfd92d300ad159887dc"), "nro_legajo" : 3, "nombre" : "empleado3" }
\end{verbatim}

\subsection{Query 2}
Debemos poder conocer los articulos mas vendidos. Para ello, definimos una tabla articulo con los siguientes campos:

\begin{verbatim}
articulos{
	codigo : STRING
	nombre : STRING
	ventas : [{dni : STRING}]
}
\end{verbatim}

De esta manera, la longitus de la lista ventas nos dice cuantas veces fue vendido un articulo. Populemos la tabla de articulos con los siguientes ejemplo:

\begin{verbatim}
db.articulos.insert({
	codigo : "1",
	nombre : "articulo1",
	ventas : [{dni :	 "11111111"}, {dni : "22222222"}]
})

db.articulos.insert({
	codigo : "2",
	nombre : "articulo2",
	ventas : [{dni : "11111111"}, {dni : "33333333"}]
})

db.articulos.insert({
	codigo : "3",
	nombre : "articulo3",
	ventas : [{dni : "33333333"}]
})
\end{verbatim}

De esta manera, la siguiente query responde nuestra consulta:

\begin{verbatim}
db.articulos.aggregate([
	{$project : {"cant_ventas" : {$size : "$ventas"} , nombre : 1, codigo : 1}},
	{$sort : {"cant_ventas" : -1}},
])
\end{verbatim}

La respuesta a esta consulta es:

\begin{verbatim}
{ "_id" : ObjectId("56420d1e92d300ad159887de"), "codigo" : "1", "nombre" : "articulo1", "cant_ventas" : 2 }
{ "_id" : ObjectId("56420d1e92d300ad159887df"), "codigo" : "2", "nombre" : "articulo2", "cant_ventas" : 2 }
{ "_id" : ObjectId("56420d1e92d300ad159887e0"), "codigo" : "3", "nombre" : "articulo3", "cant_ventas" : 1 }
\end{verbatim}

Asi, vemos que los articulos mas vendidos son los de codigo 1 y 2, cada uno con 2 ventas.

\subsection{Query 3}
Debemos poder conocer los sectores donde trabaja exactamente 3 empleados. Para ello, definimos el documento \textbf{sectores} que tiene la informacion sobre cada sector, incluyendo los empleados que trabajan en el.

\begin{verbatim}
sectores{
	cod_sector : INTEGER
	articulos : [{codigo : STRING}]
	trabaja : [{nro_legajo : INTEGER, id_tarea : INTEGER}]
}
\end{verbatim}

Populemos la tabla de empleados:

\begin{verbatim}
db.sectores.insert({
	cod_sector : 1,
	articulos : [{codigo : "1"}],
	trabaja : [	{nro_legajo : 1, id_tarea : 1},
				{nro_legajo : 2, id_tarea : 1},
				{nro_legajo : 3, id_tarea : 1},
				{nro_legajo : 4, id_tarea : 1}]
})

db.sectores.insert({
	cod_sector : 2,
	articulos : [{codigo : "2"}],
	trabaja : [	{nro_legajo : 1, id_tarea : 1},
				{nro_legajo : 2, id_tarea : 1},
				{nro_legajo : 3, id_tarea : 1}]
})
\end{verbatim}

De esta manera, la query queda:

\begin{verbatim}
db.sectores.find(
	{trabaja : {$size : 3}},
	{cod_sector : 1}
)
\end{verbatim}

Si la corremos, obtenemos:

\begin{verbatim}
{ "_id" : ObjectId("56420d9e92d300ad159887e2"), "cod_sector" : 2 }
\end{verbatim}

Efectivamente, el sector 2 contiene exactamente 3 trabajadores.

\subsection{Query 4}
Debemos conocer el empleado que trabaja en mas sectores. Esto es facil, usando los documentos ya definidos:

\begin{verbatim}
db.empleados.aggregate([
	{$project : {cant_sectores : {$size : "$sectores_donde_trabaja"} , nombre : 1, nro_legajo : 1}},
	{$sort : {cant_sectores : -1}},
	{$limit : 1}
])
\end{verbatim}

La respuesta es:

\begin{verbatim}
{ "_id" : ObjectId("56420cfd92d300ad159887da"), "nro_legajo" : 1, "nombre" : "empleado1", "cant_sectores" : 2 }
\end{verbatim}

Si bien los empleados 2 y 3 tambien trabajan en dos sectores, solo se nos pide dar uno solo, asi que desempatamos arbitrariamente.

\subsection{Query 5}
Esta query pide el ranking de los clientes con mayor cantidad de compras. Para ello, definimos los documentos \textbf{cliente}, que tienen los siguientes atributos:

\begin{verbatim}
clientes{
	dni : STRING
	nombre : STRING,
	edad : INTEGER,
	atendido_por : [{nro_legajo : INTEGER, fecha : DATETIME}],
	compro_articulos : [{codigo : INTEGER}]
}
\begin{verbatim}

Insertamos algunos clientes:

\begin{verbatim}
db.clientes.insert({
	dni : "11111111",
	nombre : "cliente1",
	edad : 18,
	atendido_por : [{nro_legajo : 1, fecha : "01/01/2015"}],
	compro_articulos : [{codigo : "1"} , {codigo : "2"}]
})

db.clientes.insert({
	dni : "22222222",
	nombre : "cliente2",
	edad : 17,
	atendido_por : [{nro_legajo : 1, fecha : "01/02/2015"} , {nro_legajo : 2, fecha : "02/02/2015"}],
	compro_articulos : [{codigo : "1"}]
})

db.clientes.insert({
	dni : "33333333",
	nombre : "cliente3",
	edad : 19,
	atendido_por : [{nro_legajo : 1, fecha : "01/03/2015"}],
	compro_articulos : [{codigo : "2"} , {codigo : "3"}]
})
\end{verbatim}

La query queda:

\begin{verbatim}
db.clientes.aggregate([
	{$project : {cant_compras : {$size : "$compro_articulos"} , nombre : 1, dni : 1}},
	{$sort : {cant_compras : -1}}
])
\end{verbatim}

El resultado es:

\begin{verbatim}
{ "_id" : ObjectId("56420df292d300ad159887e3"), "dni" : "11111111", "nombre" : "cliente1", "cant_compras" : 2 }
{ "_id" : ObjectId("56420df392d300ad159887e5"), "dni" : "33333333", "nombre" : "cliente3", "cant_compras" : 2 }
{ "_id" : ObjectId("56420df292d300ad159887e4"), "dni" : "22222222", "nombre" : "cliente2", "cant_compras" : 1 }
\end{verbatim}

Como vemos, los clientes aparecen ordenamos de mayor a menos cantidad de compras.

\subsection{Query 6}
Nos piden la cantidad de compras realizadas por clientes de misma edad. Esta query puede ser respondida con los documentos ya creados anteriormente facilmente:

\begin{verbatim}
db.clientes.aggregate([
	{$project : {cant_compras : {$size : "$compro_articulos"} , nombre : 1, dni : 1, edad : 1}},
	{$group :{
		_id : "$edad",
		total_compras : {$sum : "$cant_compras"}
	}}
])
\end{verbatim}

El resultado de la query es:

\begin{verbatim}
{ "_id" : 19, "total_compras" : 2 }
{ "_id" : 17, "total_compras" : 1 }
{ "_id" : 18, "total_compras" : 2 }
\end{verbatim}

De esta manera, creamos un diseño de documentos que, mediante documentos adecuados y redundancia, permite responder rapidamente a las queries pedidas.
